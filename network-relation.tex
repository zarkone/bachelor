\subsection{Сетевое взаимодействие компонентов системы}

Сервер и клиенты общаются между собой по сети. Когда в сети появляется
новый клиент, он подписывается на интересующие его события
проигрывателя, такие как:

\begin{itemize}
\item{ началось воспроизведение новой композиции }
\item{ сменился список воспроизведения }
\item{ сменилась временная позиция }
\end{itemize}

Таким образом, взаимодействие между клиентами и сервером происходит по
принципу \textit{publish/subscribe}.

% \addimghere{publish-subscribe}{0.6}{Взамодействие по протоколу публикация/подписка}

Когда клиент только появляется в сети или запрашивает данные,
взаимодействие между клиентом и сервером осуществляется по принципу
\textit{socket} ---  двусторонняя связь между новым клиентом и
сервером.

По команде \textit{getCommandList} или при запросе несуществующей
команды сервер возвращает код ошибки и список доступных команд. Такое
архитектурное решение было выбрано с целью обеспечить возможность
легкой расширяемости системы команд сервера. С помощью списка команд
клиент узнаёт о возможностях сервера и может решить, какие функции
будут доступны в интерфейсе, а какие нужно убрать. Например, если
сервер не отправляет информации о текущей временной позиции, то
зависящие от неё элементы можно исключить или выдать сообщение, что
запуск клиента не возможен. В то же время простым клиентам, которые
могут только переключить композицию на следующий эта функция вообще не
известна, они работают и без неё --- им нужно лишь информация о
доступности команды \textit{next} на сервере.
