\subsection{Контроллер }

Контроллер --- это <<ядро>> плеера, его серверная часть. Именно он проигрывает музыку и
отвечает на запросы от клиентов. Контроллер предоставляет доступ к
сервисам модели и интерфейсы управления для клиентов. При запуске
контроллер создает точку подключения и ожидает запросы клиентов.
Контроллер реализован как традиционный процесс-демон (Daemon)(\textit{Daemon}).  При
запуске контроллер создает точку подключения и ожидает запросы клиентов.

К функциям контроллера относятся обработка запросов клиентов
(проиграть трек, пауза) и подписка клиентов на события(сменился трек,
трек приостановлен). Клиент подключается по протоколу
\textit{tcp}. Это может быть соединение на основе запроса/ответа или
постоянное подключение.

% \addimghere{controller.eps}{0.5}{Контроллер}

Для иерархических связей в сервере используется каноничная для
\textit{JavaScript} прототипная модель.  Прототипное программирование
--- стиль объектно-ориентированного программирования, при котором
отсутствует понятие класса, а наследование производится путём
клонирования существующего экземпляра объекта --- прототипа\cite{wiki:prototype}.

Для создания сервера используется базовый прототип в котором реализованы функции
для формирования сообщений, их обработки и отправки. При создании экземпляра
сервера базовый прототип расширяется массивом команд, которые он может
обрабатывать.
