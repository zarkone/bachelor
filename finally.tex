\anonsection{Заключение}

Таким образом, представленная система не только успешно выполняет свои
задачи, но также выгодно отличается от аналогов клиент-серверной
расширяемой архитектурой на основе паттерна MVC и возможностью
трансляции комментариев к различным частям композиции.

На основе кода базовой серверной компоненты можно создавать различные медиа-сервера, меняя лишь
модель и корректируя по необходимости функции обработки потоков
данных.

Разработанная архитектура позволяет создавать композиты из серверов для интеграции
плей-листов с нескольких онлайн-сервисов и локальных коллекций одновременно.

Связь между сервером и клиентами ведется посредством
текстового обмена данными, что полностью соответствует принципам
философии UNIX. Это позволяет создавать различные клиенты с очень
простой архитектурой и очень большими возможностями для интеграции.

Работа обсуждена на конференции <<Наука и молодежь-2014>> и опубликована
в материалах конференции.\cite{secna:nim}

