\subsection{Представление}

Представление является расширяемой частью используемой архитектуру. В проекте
+реализован консольный клиент к разработанному серверу.

Клиентское приложение может использовать весь функционал сервера, либо только часть
доступных функций. Для интеграции различных расширений сервера и клиента
используется следующий подход. Клиент запрашивает список команд,
доступных на сервере(getCommandList). Получив список команд клиент
оценивает набор функций, который сможет предоставить 
пользователю и принимает решение о запуске своих модулей, отображении
элементов интерфейса и о запуске приложения в целом.

Клиент может подписаться на интересующие его события, а может лишь выполнить одну
команду и завершить работу. Клиент может и не требовать никаких команд
вовсе, лишь принять список команд и выполнить команду пользователя из
командной строки. Это самый простой тип клиентов, представляющих из
себя <<адаптер>> текстового интерфейса доступа к контроллеру.

% Для облегчения создания таких клиентов удобно иметь библиотеку,
% реализующую запросы к контроллеру для нужного клиенту
% протокола. Например, библиотека REST over HTTP, WebSocket,
% IPC и пр. Также, для разработки приложений могут использоваться
% различные языки программирования. Поэтому подразумевается, что каждый
% модуль, реализающий абстракцию протокола, имеет в комплекте поставки
% также и реализацию клиентской библиотеки.  Такая библиотека служит
% <<мостом>> для приложений--клиентов, позволяя разработчику
% сконцентрироваться на программировании логики и поведения интерфейса,
% не отвлекаясь на реализацию взаимодействия.



В ходе работы был реализован клиент с текстовым интерфейсом. Программа
клиента состоит из описания компонентов интерфейса и функций обработки
ввода пользователя. При запуске или изменении размеров окна эмулятора
терминала производится расчет размеров, компоновка компонентов
интерфейса на экране и инициализация обработчиков событий
пользовательского ввода. Обработчики событий могут отправить серверу
команду, активировать или спрятать элементы интерфейса.

Также есть обработчики сообщений сервера и объект запроса. При
инициализации обработчики подписываются на интересующие клиент
оповещения от сервера: переключение композиции, смена статуса
воспроизведения, изменение текущей временной отметки проигрываемой
композиции и обновляют соответствующие элементы интерфейса по мере их
поступления. Объект запроса посылает текстовые команды серверу и
обрабатывает ответ. Для обработки и оповещений и запросов используются одни и
те же функции, так как получаемые сообщения в обоих случаях также идентичны, как и
логика их обработки.


\addimghere{client-server-activity}{0.75}{Диаграмма деятельности клиента}

Несколько слов о реализации вывода комментариев. Проигрыватель
воспроизводит композиции из сети интернет. Соединение с интернетом не
всегда стабильно, поэтому необходимо четко контролировать процесс
вывода комментария в точности в тот момент, когда он был написан, ---
это является ключевой функцией приложения.

В клиент-серверном плеере MPD нет необходимости выводить комментарии и музыка проигрывается с
локальных носителей. Поэтому в клиентах к MPD нет синхронизации
времени с сервером, подсчет ведется внутри клиента и синхронизация
происходит только если произойдет событие смены композиции, остановки
или начала воспроизведения. В комментарии к исходным файлам клиента
\textit{Ncmpcpp} эта функция называется <<режим Idle>>. За счет сетевой
передачи сигнала могут происходить небольшие смещения, но это не
критично для пользователя. Композиции хранятся на локальном диске,
поэтому вероятность задержек воспроизведения очень мала.

В случае веб-стриминга все наоборот. Экспериментальным путем
выяснено, что частота синхронизации текущей позиции в треке на клиенте
и сервере три раза в секунду достаточна для предотвращения
<<убегания>> комментариев вперед или заметных задержек. Такой подход
создает нагрузку на сеть, но она не существенна для локальных сетей,
на работу в которых расчитано приложение, в то время как алгоритм
реализации остается простым и понятным, а пользовательский
интерфейс комментариев ведет себя в соответсвии с ожиданиями
пользователя.
