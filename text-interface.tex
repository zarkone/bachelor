\subsection{Текстовые интерфейсы}

Консольные приложения --- одна из важнейших вех в истории развития
компьютеров. До появления графических интерфейсов консоль была
единственным средством общения с ЭВМ и даже сейчас, в эпоху
победившего GUI, или графического интерфейса, консоль никуда не исчезла. Конечно, консольные
приложения, как правило, не обладают таким красивым и приятным для
глаз интерфейсом.  Не всегда сразу понятно как ими пользоваться,
особенно для неискушенных пользователей, --- приложения с GUI более
очевидны для восприятия, с GUI легко начать работать. Однако
пользователи, освоившие текстовый интерфейс, выполняют задачи гораздо
быстрее, так как не тратят время на поиск пунктов меню, не
сосредотачиваются на <<прицеливании>> в мелкие элементы интерфейса, их
внимание не рассеивается на графические декорации
\cite{infoworld:gui-tui}. Операция в текстовом интерфейсе представляет
из себя нажатие одной или нескольких клавиш, интерфейс минималистичен
и содержит в себе лишь необходимые элементы, позволяя сосредоточится
на поставленной задачи и выполнить её максимально быстро.

Одной из мощнейших возможностей программ с текстовым интерфейсом
является возможность объединять несколько программ в
конвейер. Конвейер \cite{wiki:pipe}
в терминологии UNIX --- некоторое множество процессов, для которых
выполнено следующее перенаправление ввода-вывода: то, что выводит на
поток стандартного вывода предыдущий процесс, попадает в поток
стандартного ввода следующего процесса.

Механизм конвейеров добавляет гибкость процессу взаимодействия
программ.  Все программы оперируют текстовыми данными, т.е. получают
на вход текст и выводят его же, причем, обработка данных внутри
программы идет <<в потоке>>. Данные как-бы <<протекают>> через цепочку
программ, разделенных символом конвейера в командной строке, при этом
обрабатываемый текст не копируется каждый раз от операции к операции
--- данные модифицируются <<на месте>>. Таким образом, имея некоторый набор
простых утилит для обработки текста в потоке, пользователь может
<<собрать>> себе новую программу под свои задачи, просто объединив
конвейером несколько программ с помощью символа |

\begin{bashcode}
  tail -f /var/log/Xorg.0.log | grep EE
\end{bashcode}

Кроме того, программы с текстовым интерфейсом очень быстро работают по
сети, то есть в случае, когда приложение запускается на удаленном
сервере и изменения интерфейса передаются по сети. Для работы с
интерфейсом GUI по каналам связи необходимо передавать растровую
графику. Объём информации для передачи уменьшится на несколько
порядков, если воспользоваться текстовым интерфейсом. Кроме того,
текст гораздо лучше поддается сжатию.

Таким образом, CLI-интерфейс, или интерфейс командной строки, является
одним из наиболее гибких средств 
для создания прикладных, и, в особенности, служебных приложений, а
также предоставляет удобнейшее средство для применения модульного
подхода к построению архитектуры программных систем.

Такой интерфейс приложений особо востребован среди пользователей *nix
систем. Поэтому создание консольного проигрывателя для
\textit{Soundcloud} для ОС Linux является актуальной задачей.
