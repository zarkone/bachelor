\subsection{ Сетевое взаимодействие}

Сервер и клиенты общаются между собой по сети. Когда в сети появляется
новый клиент он подписывается на интересующие его события
проигрывателя, такие как: начал проигрываться новый трек, сменился
список воспроизведения, проигрывание остановлено или возобновлено и
т.д. Такой протокол взаимодействия называется
\textit{publish/subscribe}.

Когда клиент только появляется в сети или запрашивает данные не меняя
текущего статуса проигрывателя, взаимодействие между клиентом и
сервером осуществляется по протоколу \textit{socket} ---
то есть создается двустороннее соединение и сообщения от сервера получает
только клиент, инициализировавший передачу данных.

По команде \textit{getCommandList} или при запросе несуществующей
команды сервер возвращает код ошибки и список доступных команд. Такое
архитектурное решение было выбрано с целью обеспечить расширяемость
сервера. С помощью списка команд клиент узнаёт о возможностях сервера
и может решить, какие функции будут доступны в интерфейсе, а какие
нужноубрать. Например, если сервер не отправляет информации о текущей
временной позиции, то зависящие от неё элементы можно исключить или
выдать сообщение, что запуск клиента не возможен. В то же время
простым клиентам, которые могут только переключить трек на следующий
эта функция вообще не известна, они работают и без нее --- им нужно
лишь знать, есть ли команда Next и т.д.
