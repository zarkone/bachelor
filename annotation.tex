\anonsection{Аннотация}

В данной работе спроектирована архитектура приложения c уклоном
на организацию удобного доступа и манипуляции с данными из распределенных источников. Приложение
должно быть легко адаптируемым для различных прикладных задач и, в
идеале,  представляет собой некий <<скелет>>, на базе которого
можно создать как небольшую утилиту, так и программный продукт средних
размеров. Арихтектура такого приложения не зависит от
конкретных реализаций ее компонентов --- такой подход считается наиболее
приемлемым при разработке программного обеспечения общего назначения,
его же будем придерживаться и здесь.

Данные, с которыми работает приложение никак не повлияют на
архитектуру. Иначе говоря, не должно быть никакой разницы какие
конкретно  данные или сервисы предоставляет конечный продукт --- это
лишь детали реализации конкретных клиентов (и, возможно,
отдельных сервисов внутри модели). В итоге должно получится нечто вроде
<<заготовки>> для будущего приложения.

Архитектура MVC(Model-View-Controller) и производные от нее подходы
(MVP, MVVM, далее: MV*) являются  общепринятыми при построении систем такого
класса. Часто применение архитектуры MV* на практике можно встретить в
архитектуре сложных систем приложений и программных фреймворках.

Более всего она напоминает MVVM
(Model-View-ViewModel) архитектуру, чаще всего использующуюся в
приложениях, где необходимо связывание дынных (Data Binding) модели и элементов
управления из представления. Подробнее отклонения от каноничных 
подходов будут освещены при описании конкретных элементов архитектуры.

