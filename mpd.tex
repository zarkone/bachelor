\subsection{Медиа-сервер MPD}

{\itshape Music Player Daemon}\cite{mpd:wiki} --- музыкальный
проигрыватель с клиент-серверной архитектурой. Предназначен для
воспроизведения музыкальных файлов, расположенных на локальных
носителях. Серверная часть воспроизводит музыку и отвечает на запросы
от клиентов, а клиенты управляют воспроизведением и отображают текущий
статус проигрывателя. В настоящий момент реализовано множество
клиентов, реализующих различные парадигмы интерфейса под множество
платформ. На сегодняшний день наиболее продвинутым текстовым клиентом
является {\itshape ncmpcpp} \cite{ncmpcpp:main}.

\addimghere{ncmpcpp}{0.8}{Плейлист ncmpcpp}

{\itshape Ncmpcpp} поддерживает большинство возможностей {\itshape
MPD}, имеет логичное устройство внутренних разделов и интуитивно
понятный интерфейс. По умолчанию поддерживаются сочетания клавиш
{\itshape Vim} и {\itshape Emacs}, являющиеся традиционными для
текстовых интерфейсов.

Поддержка \textit{Soundcloud} для \textit{MPD} реализована в виде
неофициального патча от одного из
пользователей\cite{mpd:sc-patch}. Данный патч добавляет возможность
проигрывать треки путем отправки на сервер команды следующего вида:
\begin{bashcode}
  mpc load soundcloud://track/≤track-id>
\end{bashcode}

К сожалению, других возможностей \textit{Soundcloud}, включая
возможность просмотра комментариев к композициям, этот патч не
предусматривает. Более того, для реализации этих функций необходимы не
только изменения кода в серверной компоненте \textit{MPD}, но также и
реализация отдельного клиента с поддержкой этих функций.
